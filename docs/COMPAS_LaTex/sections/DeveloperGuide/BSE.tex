\subsection{Binary Star Evolution}\label{sec:BinaryStarEvolution}

\subsubsection{Class Hierarchy}\label{sec:BSEClassHierarchy}

The main class for binary star evolution is the \textbf{BinaryStar} class.  The BinaryStar class is a wrapper that abstracts away the details of the binary star and the evolution.  Internally the BinaryStar class maintains a pointer to an object representing the binary star being evolved, with that object being an instance of the \textbf{BaseBinaryStar} class.

The BaseBinaryStar class is the main class for the underlying binary star object held by the BinaryStar class.  The BaseBinaryStar class defines all member variables that pertain specifically to a binary star, and many member functions that provide binary-star specific functionality.  Internally, the BaseBinaryStar class maintains pointers to the two \textbf{BinaryConstituentStar} class objects that constitute the binary star.

The BinaryConstituentStar class inherits from the Star class, so objects instantiated from the BinaryConstituentStar class inherit the characteristics of the Star class, particularly the stellar evolution model.  The BinaryConstituentStar class defines member variables and functions that pertain specifically to a constituent star of a binary system but that do not (generally) pertain to single stars that are not part of a binary system (there are some functions that are defined in the BaseStar class and its derived classes that deal with binary star attributes and behaviour -- in some cases the stellar attributes that are required to make these calculations reside in the BaseStar class so it is easier and cleaner to define the functions there).

The inheritance chain is as follows:

BinaryStar\ \rarr\ BaseBinaryStar

\tabto{2em}(Star\ \rarr\ )\tabto{7em}BinaryConstituentStar (star1) \\
\tabto{2em}(Star\ \rarr\ )\tabto{7em}BinaryConstituentStar (star2)


\newpage
\subsubsection{Evolution Model}\label{sec:BSE_EvolutionModel} 

The binary evolution model is driven by the \textbf{Evolve()} function in the BaseBinaryStar class, which evolves the star through its entire lifetime by doing the following:

\bigskip
if touching \\
\tabto{3em} STOP = true \\
else \\
\tabto{3em}calculate initial time step \\
\tabto{3em}STOP = false

\medskip
DO WHILE NOT STOP AND NOT max iterations:

\tabto{3em}evolve a single time step \\
\tabto{5em}evolve each constituent star a single time step (see SSE evolution)

\tabto{3em}if error OR unbound OR touching OR Massless Remnant \\
\tabto{5em}STOP = true \\
\tabto{3em}else \\
\tabto{5em}evaluate the binary

\tabto{7em}calculate mass transfer \\
\tabto{7em}calculate winds mass loss

\tabto{7em}if common envelope \\
\tabto{9em}resolve common envelope \\
\tabto{7em}else if supernova \\
\tabto{9em}resolve supernova \\
\tabto{7em}else \\
\tabto{9em}resolve mass changes

\tabto{7em}evaluate supernovae

\tabto{7em}calculate total energy and angular momentum \\
\tabto{7em}update magnetic field and spin: both constituent stars

\medskip
\tabto{5em}if unbound OR touching OR merger \\
\tabto{7em}STOP = true \\
\tabto{5em}else \\
\tabto{7em}if NS+BH \\
\tabto{9em}resolve coalescence

\tabto{9em}if AIS exploratory phase \\
\tabto{11em}calculate DCO Hit

\tabto{9em}STOP = true \\
\tabto{7em}else \\
\tabto{9em}if WD+WD OR max time \\
\tabto{11em}STOP = true \\
\tabto{9em}else \\
\tabto{11em}if NOT max iterations \\
\tabto{13em}calculate new time step
