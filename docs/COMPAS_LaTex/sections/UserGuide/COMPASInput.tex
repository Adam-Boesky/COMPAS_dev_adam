\subsection{COMPAS Input}\label{sec:COMPASInput}

COMPAS provides wide-ranging functionality and affords users much flexibility in determining how the synthesis and evolution of stars (single or binary) is conducted.  Users configure COMPAS functionality and provide initial conditions via the use of program options and grid files.

\subsubsection{Program Options}\label{sec:ProgramOptions}

COMPAS provides a rich set of configuration parameters via program options, allowing users to vary many parameters that affect the evolution of single and binary stars, and the composition of the population of stars being evolved. Furthermore, COMPAS allows some parameters to be specified as ranges, or sets, of values via the program options, allowing users to specify a grid of parameter values on the commandline.  Combining commandline program options ranges and sets with a grid file allows users more flexibility and the ability to specify more complex combinations of parameter values.

Not all program options can be specified as ranges or sets of values.  Options for which mixing different values in a single execution of COMPAS would either not be meaningful or might cause undesirable results, such as options that specify the mode of evolution (e.g. \texttt{-{}-}\textit{mode}), or the name or path of output files (e.g. \texttt{-{}-}\textit{output-path}, \texttt{-{}-}\textit{logfile-detailed-output} etc.), cannot be specified as a range or set of values. COMPAS will issue an error message if ranges or sets are specified for options for which they are not supported.

The full list of program options provided by COMPAS is described in Appendix~\crossref{sec:ProgramOptionsAppendix}. 

\paragraph{Program Option Ranges}\label{sec:ProgramOptionsRanges}\mbox{}

A range of values can be specified for any numeric options (i.e. integer (or integer variant), and floating point (or floating point variant) data types) that are not excluded from range specifications (see note above).

Option value ranges are specified by

\tabto{3em}\texttt{-{}-}\textit{option-name}~\textit{range-specifier}

where \textit{range-specifier} is

\tabto{3em}\textit{range-identifier}[\textit{start},\textit{count},\textit{increment}]

and 

\textit{range-identifier}\tabto{7em}is one of \lcb'r', 'range'\rcb~(case is not significant)\\
\textit{start}\tabto{7em}is the starting value of the range\\
\textit{count}\tabto{7em}is the number of values in the range (must be an \textit{unsigned long int})\\
\textit{increment}\tabto{7em}is the amount by which the value increments for each value in the range

Note that

\textit{range-identifier} is optional for \textit{range-specifier}.\\
\textit{start} and \textit{increment} must be the same data type as \textit{option-name}.\\
\textit{count} must be a positive integer value.

To specify a range of values for the \textit{ \texttt{-{}-}metallicity} option, a user, if running COMPAS from the commandline and with no grid file, would type any of the following:

\tabto{3em}./COMPAS \texttt{-{}-}metallicity [0.0001,5,0.0013]
\tabto{3em}./COMPAS \texttt{-{}-}metallicity r[0.0001,5,0.0013]
\tabto{3em}./COMPAS \texttt{-{}-}metallicity range[0.0001,5,0.0013]

In each of the examples above the user has specified, by the use of the \textit{range-specifier}, that five binary stars should be evolved, with constituent star metallicities = 0.0001, 0.0014, 0.0027, 0.0040, and 0.0053.

To evolve a grid of binaries with ten different metallicities, starting at 0.0001 and incrementing by 0.0002, and five different common envelope alpha values, starting at 0.1 and incrementing by 0.2, the user would type

\tabto{3em}./COMPAS \texttt{-{}-}metallicity [0.0001,10,0.0013] \texttt{-{}-}common-envelope-alpha [0.1,5,0.2]

and COMPAS would evolve a grid of 50 binaries using the 10 metallicity values and 5 common envelope alpha values.

\paragraph{Program Option Sets}\label{sec:ProgramOptionsSets}\mbox{}

A set of values can be specified for options of any data type that are not excluded from set specifications (see note above).

Option value sets are specified by

\tabto{3em}\texttt{-{}-}\textit{option-name}~\textit{set-specifier}

where \textit{set-specifier} is

\tabto{3em}\textit{set-identifier}[\textit{value$_1$},\textit{value$_2$},\textit{value$_3$},\,...\,,\textit{value$_n$}]

and 

\textit{set-identifier}\tabto{7em}is one of \lcb's', 'set'\rcb~(case is not significant)\\
\textit{value$_i$}\tabto{7em}is a value for the option\\

Note that

\textit{set-identifier} is mandatory for \textit{set-specifier}.\\
\textit{value$_i$} must be the same data type as \textit{option-name}.\\
Valid values for boolean options are \lcb 1$\vert$0, TRUE$\vert$FALSE, YES$\vert$NO, ON$\vert$OFF\rcb, and all set values must be of the same type (i.e. all 1$\vert$0, or all YES$\vert$NO etc.).\\
There is no limit to the number of values specified for a set, values can be repeated, and neither order nor case is significant.

To specify a set of values for the \textit{\texttt{-{}-}eccentricity-distribution} option, a user, if running COMPAS from the commandline and with no grid file, would type any of the following:

\tabto{3em}./COMPAS \texttt{-{}-}eccentricity-distribution s[THERMALISED,FIXED,FLAT]\\
\tabto{3em}./COMPAS \texttt{-{}-}eccentricity-distribution set[THERMALISED,FIXED,FLAT]

In each of the examples above the user has specified, by the use of the \textit{set-specifier}, that three binary stars should be evolved, using the eccentricity distributions 'THERMALISED', 'FIXED', and 'FLAT'.

\subsubsection{Grid Files}\label{sec:GridFiles}

A grid file allows users to specify initial values for multiple systems for both Single Star Evolution (SSE) and Binary Star Evolution (BSE) for: each line of a grid file is used by COMPAS to set the initial values of an individual single star (SSE) or binary star (BSE), and each single star or binary star defined by a grid file line is evolved using those initial values.

Each line of a grid file is a set of option specifications, with the specifications being exactly as they would appear on the commandline if running COMPAS from the commandline.

For example, a grid file could contain the following two lines:

\tabto{3em}\texttt{-{}-}metallicity 0.001 \texttt{-{}-}eccentricity 0.0 \texttt{-{}-}remnant-mass-prescription fryer2012\\
\tabto{3em}\texttt{-{}-}remnant-mass-prescription mullermandel \texttt{-{}-}metallicity 0.02 \texttt{-{}-}semi-major-axis 45.678

in which case COMPAS would evolve two binaries, with the option values set per the grid file lines.

\bigskip
Grid files can have blank lines and comments.  Comments begin with a hash/pound character (‘\#’) - the hash/pound character and text beyond it are ignored by COMPAS.

Not all program options can be specified in a grid file.  Options that should remain constant for a single execution of COMPAS, such as options that specify the mode of evolution (e.g. \texttt{-{}-}\textit{mode}), or the name or path of output files (e.g. \texttt{-{}-}\textit{output-path}, \texttt{-{}-}\textit{logfile-detailed-output} etc.) can only be specified on the commandline. COMPAS will issue an error message if an option that is not supported in a grid file is specified on a grid file line.

\paragraph{Program Option Defaults}\label{sec:ProgramOptionsdefaults}\mbox{}

Any program options that are not specified  take default values:

\begin{itemize}
\item On a grid file line, program options that are not explicitly specified default to the value specified for that option on the commandline.
\item On the commandline, program options that are not explicitly specified default to the COMPAS default value for the option (as specified in the COMPAS code - may be sampled from a distribution).
\end{itemize}

This means that a program option not explicitly specified on a grid file line will take the value for that option as it was specified on the commandline, or the COMPAS default value if the option was not explicitly specified on the commandline. That is, the value for any option not specified on a grid file line option falls back to the value specified on the commandline, which falls back to the COMPAS default if it was not specified on the commandline.

\paragraph{Mixing Ranges and Sets}\label{sec:MixingRangesAndSets}\mbox{}

Ranges and sets can be specified together, and there is no limit to the number of ranges or sets that can be specified on the commandline, or in the grid file.

\bigskip
Consider the following grid file, named 'gridfile.txt':

\tabto{3em}\texttt{-{}-}metallicity r[0.0001,5,0.0013] \texttt{-{}-}common-envelope-alpha s[0.1,0.2,0.6,0.9]
\tabto{3em}\texttt{-{}-}fryer-supernova-engine s[rapid,delayed] \texttt{-{}-}eccentricity r[0.0001,3,0.0003]

\bigskip
Running COMPAS with

\tabto{3em}./COMPAS \texttt{-{}-}grid gridfile.txt

\bigskip
would result in 26 binaries being evolved:
\begin{itemize}
\item 20 for the first grid line (5 for the range of metallicities, times 4 for the set of CE alpha values), and
\item 6 for the second grid line (2 for the set of Fryer SN engine values, and 3 for the range of eccentricities)
\end{itemize}

\bigskip
In the example above, running COMPAS with

\small
\tabto{2.5em}./COMPAS \texttt{-{}-}remnant-mass-prescription s[mullermandel,fryer2012,hurley2000,muller2016] \texttt{-{}-}grid gridfile.txt
\normalsize

\bigskip
would result in 104 binaries being evolved: the grid file would be ‘executed’ for each of the four remnant mass prescriptions specified on the commandline.

Multiple ranges and/or sets can be specified on the commandline, and on each line of the grid file – so very large numbers of stars/binaries can be evolved with just a few range/set specifications.

\paragraph{Specifying initial random seed values}\label{sec:RandomSeed}\mbox{}

The \textit{\texttt{-{}-}random-seed} option allows users to specify the initial value to be used to seed the pseudo-random number generator. Once set, the random seed values increments from its initial value for each star, or binary star, evolved. How the random seed increments depends upon the context.

The \textit{\texttt{-{}-}random-seed} option can be specified on either, or both, the commandline and a grid file line. If the option is not specified on one or the other, it defaults per the description above (see Section~\crossref{sec:ProgramOptionsdefaults}).

In general, if the \textit{\texttt{-{}-}random-seed} option is specified, the pseudo-random number generator will be seeded using the specified value for the first star, or binary star, evolved, then for each subsequent star or binary star, the seed value will be incremented by one and the pseudo-random number generator re-seeded. Seeding the pseudo-random number generator with a known seed for each star, or binary star, evolved ensures that the evolution of specific stars, or binary stars, can be reproduced.

Consider a single execution of COMPAS effected with

\tabto{3em}./COMPAS \texttt{-{}-}random-seed 15 \texttt{-{}-}number-of-systems 100 \texttt{-{}-}metallicity 0.015

\bigskip
This would evolve 100 binary stars, each with metallicity = 0.015, and other initial attributes set to their defaults. The first of the 100 binary stars will be evolved using the random seed 15, the second 16, the third 17, and so on - each binary star will evolve using a unique random seed.

In the example shown above (see Section~\crossref{sec:MixingRangesAndSets}), all 104 binary stars would evolve with unique random seed values, ranging from 0 (the default, since the option was not specified on either the commandline or in the grid file), to 103.

In both these examples, the random seed was incremented in the context of the commandline.  In the first example, the random seed was explicitly specified on the commandline, and in the second example the random seed defaulted to the commandline default.

Consider now a single execution of COMPAS, using the grid file 'mygrid.txt':

\tabto{3em}./COMPAS \texttt{-{}-}random-seed 12 \texttt{-{}-}grid mygrid.txt

\bigskip
where the contents of the grid file 'mygrid.txt' are

\tabto{3em}\texttt{-{}-}allow-rlof-at-birth true \texttt{-{}-}metallicity 0.1
\tabto{3em}\texttt{-{}-}semi-major-axis 23.4 \texttt{-{}-}random-seed 107
\tabto{3em}\texttt{-{}-}random-seed 63 \texttt{-{}-}metallicity 0.12 \texttt{-{}-}eccentricity s[0.1,0.2,0.3,0.4]
\tabto{3em}\texttt{-{}-}initial-mass-1 12.3

\bigskip
This would evolve 7 binary stars with random seed values 12, 107, 63, 64, 65, 66, and 18.

The first binary star evolved is the first line of the grid file. This line does not specify the \textit{\texttt{-{}-}random-seed} option, so the random seed defaults to the commandline value. The commandline did specify a value for the random seed (12), so that value is used. Since the first line of the grid file is the first binary star evolved, the random seed is not incremented, and the value of 12 is used.

The second binary star evolved is the second line of the grid file. This line does specify the \textit{\texttt{-{}-}random-seed} option. Since this is the first binary star evolved in the context of the random seed specified on the grid file line, the random seed is not incremented, and the value of 107 is used.

The third binary star evolved is the third line of the grid file. This line does specify the \textit{\texttt{-{}-}random-seed} option. Since this is the first binary star evolved in the context of the random seed specified on the grid file line, the random seed is not incremented, and the value of 63 is used.

The fourth, fifth, and sixth binary stars evolved are also from the third line of the grid file - a set of four values for \textit{eccentricity} was specified. Since these are subsequent to the first binary star evolved in the context of the random seed specified on the grid file line, the random seed is incremented, and the values of 64, 65, and 66 are used.

The seventh binary star evolved is the fourth line of the grid file. This line does not specify the \textit{\texttt{-{}-}random-seed} option, so the random seed defaults to the commandline value. The commandline did specify a value for the random seed (12), so that value is used, but since this binary star is subsequent to the first binary star evolved in the context of the random seed specified on the commandline, the random seed is incremented. This is the sixth subsequent binary star evolved in the context of the commandline (all stars, or binary stars, evolved in a single execution of COMPAS are evolved in the context of the commandline), so  the random seed is incremented from 12 to 18 (by 1 for each binary star evolved), and the value used for this binary star is 18.

Note that in this example, all binary stars were evolved using a unique random seed. This is because the values specified for the random seeds via the \textit{\texttt{-{}-}random-seed} option were 'well-behaved'. Unfortunately there is no reasonable way to protect the user against specifying duplicate random seeds – especially since the random see increments for each star or binary star. If the user chooses to specify multiple grid file lines with the same random seed, or initial random seeds that would collide with other random seed values and cause duplicates as they increment through ranges and sets, then there will be duplicate random seeds in the output files. Users should take care when specifying random seeds in grid files via the \textit{\texttt{-{}-}random-seed} option.

\paragraph{Running COMPAS via Python}\label{sec:PythonSubmit}\mbox{}

A convenient method of managing the many program options provided by COMPAS is to run COMPAS via Python, using a script to manage and specify the values of the program options.

Ranges and sets can be specified for options in the Python script file, but the range or set parameter must be enclosed in quotes – otherwise python tries to parse the construct. For example, to specify a set of metallicity values in the Python script file, use:

\bigskip
\tabto{3em}metallicity~=~‘s[0.001,0.002,0.003,0.007,0.01,0.015,0.02]’

\bigskip
If the set parameter is not enclosed in quotes, Python will attempt to parse it, and will fail.

\bigskip
An example Python script is provided in the COMPAS suite: \textit{pythonSubmit.py}. Users should copy this script and modify their copy to match their experimental requirements. Refer to the \textbf{Getting Started Guide} for more details.



