\subsubsection{Grids}\label{sec:Grids}

Grid functionality allows users to specify a grid of initial values for both Single Star Evolution (SSE) and Binary Star Evolution (BSE).

For SSE, users can supply a text file that contains initial mass values and, optionally, metallicity and supernova kick related values, and COMPAS will evolve individual stars with those initial values (one star per record).

For BSE, users can supply a text file that contains initial mass and metallicity values for the binary constituent stars, as well as the initial separation or orbital period, and eccentricity, of the binary (one binary star per record of initial values).

\paragraph{Grid File Format}\label{sec:GridFileFormat}\mbox{}

Grid files are comma-separated text files, with column headers denoting the meaning of the data in the column (with an exception for SSE Grid files -- see below for details).

Grid files may contain comments. Comments are denoted by the hash/pound character ('\#'). The hash character and any text following it on the line in which the hash character appears is ignored. The hash character can appear anywhere on a line - if it is the first character then the entire line is a comment and ignored, or it can follow valid characters on a line, in which case the characters before the hash are processed, but the hash character and any text following it is ignored. Blank lines are ignored.

Notwithstanding the exception for SSE Grid files mentioned above, the first non-comment, non-blank line in a Grid file must be the header record. The header record is a comma-separated list of strings that denote the meaning of the data in each of the columns in the file.

Data records follow the header record. Data records, with an exception for SSE data files described below, are comma-separated lists of non-negative floating-point numbers. Any data field that contains a negative number, or characters that do not convert to floating-point numbers, is considered an error and will cause processing of the Grid file to be abandoned -- an error message will be displayed.

Data records are expected to contain the same number of columns as the header record. If a data record contains more columns than the header record, data beyond the number of columns in the header record is ignored. If a data record contains fewer columns than the header record, missing data values (by position) are set equal to 0.0 -- a warning message will be displayed.

% BSE Grid File
\xpar{BSE Grid File}
\xpar{Header Record}        % BSE Grid File header record

The BSE Grid file header record must be a comma-separated list of strings taken from the following list (case is not significant).

% BSE Grid File table
\small
\begin{tabularx}{\linewidth}{
    |>{\hsize=0.45\hsize}X
    |>{\hsize=1.55\hsize}X
    |
}

\hline

\gridsRow{\textbf{Header string}}{\textbf{Column meaning}}

\gridsRow{\textbf{Mass\_1}}{Mass value to be assigned to the primary star ($M_{\odot}$).}

\gridsRow{\textbf{Mass\_2}}{Mass value to be assigned to the secondary star ($M_{\odot}$).}

\gridsRow{\textbf{Metallicity\_1}}{Metallicity value to be assigned to the primary star.}

\gridsRow{\textbf{Metallicity\_2}}{Metallicity value to be assigned to the secondary star.}

\gridsRow{\textbf{Separation}}{Separation of the stars -- the semi-major axis value to be assigned to the binary~(AU).}

\gridsRow{\textbf{Eccentricity}}{Eccentricity value to be assigned to the binary.}

\gridsRow{\textbf{Period}}{Orbital period value to be assigned to the binary~(days).}

\gridsRow{Kick\_Magnitude\_Random\_1 }{Value to be used as the kick magnitude random number, used to draw the kick magnitude, for the primary star should it undergo a supernova event. This must be a floating-point number in the range [0.0, 1.0). If this column is present in the grid file, the kick magnitude for the primary star, should it undergo a supernova event, will be drawn from the appropriate distribution at the time of the SN event. This column is used in preference to the \textit{Kick\_Magnitude\_1} column if both are present in the grid file.}

\gridsRow{Kick\_Magnitude\_1}{Value to be used as the (drawn) kick magnitude for the primary star should it undergo a supernova event ($km~s^{\minus1}$). If both this column and \textit{Kick\_Magnitude\_Random\_1} are present in the grid file, \textit{Kick\_Magnitude\_Random\_1} will be used in preference to \textit{Kick\_Magnitude\_1}.}

\gridsRow{Kick\_Theta\_1}{Value to be as the angle between the orbital plane and the 'z' axis of the supernova vector for the primary star should it undergo a supernova event~(radians).}

\gridsRow{Kick\_Phi\_1}{Value to be used as the angle between 'x' and 'y', both in the orbital plane of the supernova vector, for the primary star should it undergo a supernova event~(radians).}

\gridsRow{Kick\_Mean\_Anomaly\_1}{Value to be used as the mean anomaly at the instant of the supernova for the primary star should it undergo a supernova event -- should be uniform in~[0, 2$\pi$).}

\gridsRow{Kick\_Magnitude\_Random\_2}{Value to be used as the kick magnitude random number, used to draw the kick magnitude, for the secondary star should it undergo a supernova event. This must be a floating-point number in the range [0.0, 1.0). If this column is present in the grid file, the kick magnitude for the secondary star, should it undergo a supernova event, will be drawn from the appropriate distribution at the time of the SN event. This column is used in preference to the \textit{Kick\_Magnitude\_2} column if both are present in the grid file.}

\gridsRow{Kick\_Magnitude\_2}{Value to be used as the (drawn) kick magnitude for the secondary star should it undergo a supernova event ($km~s^{\minus1}$). If both this column and \textit{Kick\_Magnitude\_Random\_2} are present in the grid file, \textit{Kick\_Magnitude\_Random\_2} will be used in preference to \textit{Kick\_Magnitude\_2}.}

\gridsRow{Kick\_Theta\_2}{Value to be as the angle between the orbital plane and the 'z' axis of the supernova vector for the secondary star should it undergo a supernova event~(radians).}

\gridsRow{Kick\_Phi\_2}{Value to be used as the angle between 'x' and 'y', both in the orbital plane of the supernova vector, for the secondary star should it undergo a supernova event~(radians).}

\gridsRow{Kick\_Mean\_Anomaly\_2}{Value to be used as the mean anomaly at the instant of the supernova for the secondary star should it undergo a supernova event -- should be uniform in~[0, 2$\pi$).}

\end{tabularx}
\normalsize

\newpage
All header strings in\textbf{ bold} in the table above are required in the header record, with the exception of \textit{Separation} and \textit{Period}: one of Separation and Period \textit{must} be present, but both \textit{may} be present.

All other header strings in the table above (the kick-related header strings) are optional.  However, if one of the kick-related header strings is present, then \textit{all must} be present.

The order of the columns in the BSE Grid file is not significant.  

\xpar{Data Record}      % BSE Grid File data record

See the general description of data records above.  

As for the header record, only one of \textit{Separation} and \textit{Period} is required to be present, but both may be present. The period may be used to calculate the separation of the binary. If the separation is present it is used as the value for the semi-major axis of the binary, regardless of whether the period is present (Separation has precedence over Period). If the period is present, but separation is not, the separation is calculated form the masses of the stars and the period given.

Also as for the header record, the kick-related values are not mandatory, but if one of the kick-related values is given, then \textit{all must} be given.

% SSE Grid File
\xpar{SSE Grid File}
\xpar{Header Record}        % SSE Grid File header record

The SSE Grid file header record must be a comma-separated list of strings taken from the following list (case is not significant):

% SSE Grid File table
\small
\begin{tabularx}{\linewidth}{
    |>{\hsize=0.45\hsize}X
    |>{\hsize=1.55\hsize}X
    |
}

\hline

\gridsRow{\textbf{Header string}}{\textbf{Column meaning}}

\gridsRow{\textbf{Mass}}{Mass value to be assigned to the star~($M_{\odot}$).}

\gridsRow{Metallicity }{Metallicity value to be assigned to the star.}

\gridsRow{Kick\_Magnitude\_Random}{Value to be used as the kick magnitude random number, used to draw the kick magnitude, for the star should it undergo a supernova event. This must be a floating-point number in the range [0.0, 1.0). If this column is present in the grid file, the kick magnitude for the star, should it undergo a supernova event, will be drawn from the appropriate distribution at the time of the SN event. This column is used in preference to the \textit{Kick\_Magnitude} column if both are present in the grid file.}

\gridsRow{Kick\_Magnitude}{Value to be used as the (drawn) kick magnitude for the star should it undergo a supernova event ($km~s^{\minus1}$). If both this column and \textit{Kick\_Magnitude\_Random} are present in the grid file, \textit{Kick\_Magnitude\_Random} will be used in preference to \textit{Kick\_Magnitude}.}

\end{tabularx}
\normalsize

The SSE Grid file is only required to list Mass values for each star, with Metallicity and Kick values being optional. If the \textit{Metallicity} column is omitted, the metallicity value assigned to the star is the user-specified value for metallicity via the program option \textit{\texttt{-{}-}metallicity} (or \textit{\texttt{-{}-}z}).

If the \textit{Metallicity} column is omitted from the SSE Grid file, the header is optional: if there is only one column of data in the SSE Grid file it is assumed to be the Mass column, and no header is required (though may be present). If the \textit{Metallicity} column header is present, the \textit{Mass} column header is required.

The order of the columns in the SSE Grid file is not significant.  

\xpar{Data Record}      % SSE Grid File data record

See the general description of data records above. As for the header record, only mass is required to be present, but metallicity may also be present. If metallicity is omitted, the metallicity value assigned to the star is the user-specified value for metallicity via the program option \textit{metallicity} (or \textit{z}).
