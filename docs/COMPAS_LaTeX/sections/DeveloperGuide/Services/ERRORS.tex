\subsubsection{Error Handling}\label{sec:ErrorHandling}

An error handling service is provided encapsulated in a singleton object (an instantiation of the Errors class).

The Errors service provides global error handling functionality.  Following is a brief description of the Errors service (full documentation coming soon...):

Errors are defined in the error catalog in constants.h (see ERROR\_CATALOG).  It could be useful to move the catalog to a file so it can be changed without changing the code, or even have multiple catalogs provided for internationalisation -- a task for later.

Errors defined in the error catalog have a scope and message text.  The scope is used to determine when/if an error should be printed.

The current values for scope are:

\hfill
\begin{minipage}{\dimexpr\textwidth-2em}

        \medskip
        \begin{minipage}[t][][b]{14.0em}\textbf{NEVER}\end{minipage}
        \begin{minipage}[t][][b]{\dimexpr\textwidth-14.5em}
            the error will not be printed.
        \end{minipage}\vfill

        \medskip
        \begin{minipage}[t][][b]{14.0em}\textbf{ALWAYS}\end{minipage}
        \begin{minipage}[t][][b]{\dimexpr\textwidth-14.5em}
            the error will always be printed.
        \end{minipage}\vfill

        \medskip
        \begin{minipage}[t][][b]{14.0em}\textbf{FIRST}\end{minipage}
        \begin{minipage}[t][][b]{\dimexpr\textwidth-14.5em}
            the error will be printed only on the first time it is encountered anywhere in the program.
        \end{minipage}\vfill

        \medskip
        \begin{minipage}[t][][b]{14.0em}\textbf{FIRST\_IN\_OBJECT\_TYPE}\end{minipage}
        \begin{minipage}[t][][b]{\dimexpr\textwidth-14.5em}
            the error will be printed only on the first time it is encountered anywhere in objects of the same type (e.g. Binary Star objects).
        \end{minipage}\vfill

        \medskip
        \begin{minipage}[t][][b]{14.0em}\textbf{FIRST\_IN\_STELLAR\_TYPE}\end{minipage}
        \begin{minipage}[t][][b]{\dimexpr\textwidth-14.5em}
            the error will be printed only on the first time it is encountered anywhere in objects of the same stellar type (e.g. HeWD Star objects).
        \end{minipage}\vfill

        \medskip
        \begin{minipage}[t][][b]{14.0em}\textbf{FIRST\_IN\_OBJECT\_ID}\end{minipage}
        \begin{minipage}[t][][b]{\dimexpr\textwidth-14.5em}
            the error will be printed only on the first time it is encountered anywhere in an object instance.
        \end{minipage}\vfill

        \medskip
        \begin{minipage}[t][][b]{14.0em}\textbf{FIRST\_IN\_FUNCTION}\end{minipage}
        \begin{minipage}[t][][b]{\dimexpr\textwidth-14.5em}
            the error will be printed only on the first time it is encountered anywhere in the same function of an object instance (i.e. will print more than once if encountered in the same function name in different objects).
        \end{minipage}\vfill
\end{minipage}

\bigskip
The Errors service provides methods to print both warnings and errors -- essentially the same thing, but warning messages are prepended with "WARNING:", whereas error messages are prepended with "ERROR:".

Errors and warnings are printed by using the macros defined in ErrorsMacros.h.  They are:
 
\paragraph{\Large{Error macros}}\mbox{}

\medskip
\textbf{SHOW\_ERROR(error\_number)}
Prints "ERROR: " followed by the error message associated with \textit{error\_number} (from the error catalog).

\medskip
\textbf{SHOW\_ERROR(error\_number, error\_string)}
Prints "ERROR: " followed by the error message associated with \textit{error\_number} (from the error catalog), and appends "error\_string".

\medskip
\textbf{SHOW\_ERROR\_IF(cond, error\_number)}
If \textit{cond} is true, prints "ERROR: " followed by the error message associated with \textit{error\_number} (from the error catalog).

\medskip
\textbf{SHOW\_ERROR\_IF(cond, error\_number, error\_string)}
If \textit{cond} is TRUE, prints "ERROR: " followed by the error message associated with \textit{error\_number} (from the error catalog), and appends \textit{error\_string}.


\paragraph{\Large{Warning macros}}\mbox{}

\medskip
\textbf{SHOW\_WARN(error\_number)}
Prints "WARNING: " followed by the error message associated with \textit{error\_number} (from the error catalog).

\medskip
\textbf{SHOW\_WARN(error\_number, error\_string)}
Prints "WARNING: " followed by the error message associated with \textit{error\_number} (from the error catalog), and appends \textit{error\_string}.

\medskip
\textbf{SHOW\_WARN\_IF(cond, error\_number)}
If \textit{cond} is TRUE, prints "WARNING: " followed by the error message associated with \textit{error\_number} (from the error catalog)

\medskip
\textbf{SHOW\_WARN\_IF(cond, error\_number, error\_string)}
If \textit{cond} is TRUE, prints "WARNING: " followed by the error message associated with \textit{error\_number} (from the error catalog), and appends \textit{error\_string}

\bigskip
Error and warning message always contain:

\hfill
\begin{minipage}{\dimexpr\textwidth-2em}

    The object id of the calling object.

    The object type of the calling object.

    The stellar type of the calling object (will be "NONE" if the calling object is not a star-type  object).

    The function name of the calling function.
    
\end{minipage}

\bigskip
Any object that uses the Errors service (i.e. the SHOW\_* macros) must expose the following functions:


\tabto{3em}OBJECT\_ID  ObjectId() const \lcb\ return m\_ObjectId;\ \rcb \\
\tabto{3em}OBJECT\_TYPE ObjectType() const \lcb\ return m\_ObjectType;\ \rcb \\
\tabto{3em}STELLAR\_TYPE StellarType() const \lcb\ return m\_StellarType;\ \rcb

These functions are called by the \textbf{SHOW\_* macros}.  If any of the functions are not applicable to the object, then they must return "*::NONE (all objects should implement ObjectId() correctly).

\medskip
The filename to which error records are written when \textbf{Start()} parameter \textit{errorsToLogfile} is true is declared in constants.h -- see the LOGFILE enum class and associated descriptor map LOGFILE\_DESCRIPTOR. Currently the name is "Error\_Log".

